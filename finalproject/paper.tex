\documentclass[12pt]{article}
\title{KRACK and the Ethics of Disclosure}
\author{Adon Shapiro}

\usepackage[margin=1in, headheight=14.5pt]{geometry}
\usepackage{fancyhdr}
\usepackage{csquotes}
\usepackage{setspace}
\usepackage{enumitem}
\PassOptionsToPackage{hyphens}{url}
\usepackage[hidelinks]{hyperref}

\pagestyle{fancy}
\fancyhf{}
\renewcommand{\headrulewidth}{0pt}
\rhead{Shapiro \thepage}

\begin{document}
\maketitle
\section*{\hfil Abstract \hfil}
Vulnerabilities and exploits in software are discovered every day. This is a
well-known fact in the cyber-security industry. What is not so universally
agreed-upon is the proper way for an ethical hacker or security researcher to
disclose a vulnerability they have discovered, and the argument re-surfaces
with the disclosure of each new high-profile bug. The most recent such bug 
is KRACK.
In October 2017, researchers at the University of Leuven published details
of a vulnerability they had discovered in the WPA2 protocol that secures Wi-Fi
connections. They called it KRACK (Key Reinstallation AttaCK) because of the
way it takes advantage of a bug in WPA2 that allows the re-use of 
cryptographic keys. KRACK is significant because it is a fundamental weakness
not in a specific piece of software, but rather in the Wi-Fi standard itself,
and so everyone who uses Wi-Fi was susceptible to the attack. Because of its
recentness and its far-reaching impact, KRACK is a perfect lens through which
to view the problem of ethical disclosure.

\begin{doublespace}
\section*{Introduction}
Disclosure, as it pertains to cyber-security, refers to the manner in which
people are made aware of vulnerabilities in their software. This can mean many
different things, however. To formalize different approaches to disclosure, it
is helpful to define three roles in the process: Vendors, Customers, and
Reporters. As defined by Christey and Wysopal,
a Vendor is ``an individual or organization who provides, develops, or
maintains software, hardware, or services,'' a customer is 
an ``end user of the software, hardware, or service that may be affected by the
vulnerability,'' and a Reporter is an ``individual or organization that informs
(or attempts to inform) the Vendor of the vulnerability.''\footnote{Christey \& Wysopal, \textit{Responsible Vulnerability Disclosure Process}.}
The crux of the
issue is how different approaches to disclosure affect Customers, and the
``different methods'' refer to the actions and reactions of Reporters and
Vendors. At the extreme ends of the disclosure spectrum we have full disclosure
and non-disclosure. Full disclosure is when a Reporter makes their discovery
completely public with little to no consideration given to the consequences of
publicizing an exploit. The reasoning behind this method is generally that the
best way to get a Vendor to fix a vulnerability is to have it be as noticeable
as possible, and
that the consequences of malicious individuals learning of the attack are
negligible, as it may well have already been known by less ethical hackers.\footnote{Stephen A. Shepherd, \textit{ Vulnerability Disclosure: How do we define Responsible Disclosure}.}
Non-disclosure involves keeping the veulnerability as secret as possible. If a
Reporter even alerts a Vendor, it is unlikely either will
alert the public, sometimes even after the vulnerability has been taken
care of. The rationale being that Customers will be safest if the exploit is
kept completely secret, since no one with questionable motives will be aware
either.\footnotemark[2]
Of course, these are both extremes and in reality, most
vulnerabilities are disclosed in a manner somewhere between these two
approaches. Most bugs, KRACK included, are disclosed in manner known as
``responsible disclosure,'' which attempts to find a happy medium, but what
this actually means is also a matter of contention.

KRACK is an inherent weakness in the WPA2 protocol that
secures protected Wi-Fi networks. Essentially, the vulnerability allows an
attacker to authenticate themselves by forcing the network to reinstall an
already in-use cryptographic key (thus the name KRACK, for Key Reinstallation
AttaCK). The exploit was discovered by Belgian researchers Mathy Vanhoef and
Frank Piessens at KU Leuven in 2016 and published a paper with the results
of their findings in October 2017, however some affected Vendors were notified
earlier than this.\footnote{Catalin Cimpanu, \textit{New KRACK Attack Breaks WPA2 WiFi Protocol}.} On the spectrum of disclosure,
this is what most people would label responsible disclosure with perhaps a bit
of a bent towards full disclosure. Though some vendors were alerted before the
bug was fully disclosed, ultimately the vulnerability was made fully
known to the public, and soon after, most Vendors released patches to their
implementations of WPA2.

\section*{To the Community}
So far, it may seem that KRACK is unremarkable. It was discovered by academic
researchers, disclosed in a relatively uncontroversial manner, and quickly
patched. But it is easy to overlook the severity of the exploit. For one thing,
almost anyone who uses the internet is susceptible to it. I use the
present-tense because the fact that a patch has been released has absolutely
no bearing on whether a user has a properly updated system, especially
considering how recently the bug was disclosed. WPA2 is the standard
for secure wireless access and is relied upon for protecting almost every
modern Wi-Fi network, so the only way to be immune from KRACK without a patched
WPA2 implementation is to use a wired connection.

In terms of disclosure, KRACK's fresh discovery allows us to view the
ramifications of its disclosure in real time, and its academic source gives it
an unusual amount of legitimacy. All told, it is an interesting
opportunity to gain new perspective on the old debate.

\section*{Technical Details of the Exploit}
Since KRACK is a fundamental vulnerability in the WPA2 protocol, some
understanding of the protocol is necessary to comprehend the specifics of
a KRACK attack. Wi-Fi Protected Access II (WPA2) is the name given to
implementations of the standard for secure Wi-Fi outlined in IEEE 802.11i-2004.
The key elements of the protocol which allow encrypted traffic to flow between
a client and an access-point without transmitting the encryption key itself
are the four-way handshake and group
key handshake. First, a user must provide a pre-shared key for initial
authentication (the Wi-Fi password). A secret Pairwise Master Key (PMK) is then
generated for the session. Next, the four-way handshake verifies that both the
access point and the client have the proper PMK without needing to disclose the
key. Once this is established, two more keys are generated: the Pairwise
Transient Key (PTK) and the Group Temporal Key (GTK).\footnote{IEEE802.11i-2004.}

KRACK exploits a vulnerability in the four-way handshake process of the
protocol. The GTK is installed after the third message of the four-way
handshake, but to account for dropped or interrupted connections, WPA2 allows
the re-transmission of this message.
An attacker can intercept the encrypted traffic of the handshake between
another (legitimate) user and the access point and
need only resend the third handshake to reset the WPA2 encryption key. This is
the ``nonce reuse'' described in the title of Vanhoef's paper. Every
time the key is reset the data sent over the network will be encrypted with
the same key, allowing traffic to be gradually decrypted until the entire
key is decrypted and the network no longer secure.\footnote{Mathy Vanhoef, \textit{Key Reinstallation Attacks Breaking WPA2 by forcing nonce reuse}.}
The attack is particularly devastating against wpa\_supplicant, the open-source
Wi-Fi client typically used on Linux operating systems. This implementation,
which is also found in many Android phones, allows the attacker to simply
install an all-zero encryption key with no need at all for the real key.\footnotemark[5]

\subsection*{Defense}
Unfortunately, without a patched implementation of WPA2, there is no way to
defend against a KRACK attack. Fortunately, many vendors have released patches
that completely fix the issue. Mathy Vanhoef, the researcher who discovered the
vulnerability, provides scripts to test your machine for the vulnerability on
the website he created to provide information about the KRACK attack.

\section*{Disclosure as it Pertains to KRACK}
I said earlier that most people would consider Vanhoef's disclosure
``responsible.'' Let us now define what responsible disclosure entails. As
defined by Christey \& Wysopal, responsible disclosure aims to:
\end{doublespace}
\begin{enumerate}
  \item Ensure that vulnerabilities can be identified and eliminated
    effectively and efficiently for all parties.
  \item Minimize the risk to customers from vulnerabilities that could
   allow damage to their systems.
  \item Provide customers with sufficient information for them to evaluate
   the level of security in vendors' products.
  \item Provide the security community with the information necessary to
   develop tools and methods for identifying, managing, and reducing the
   risks of vulnerabilities in information technology.
  \item Minimize the amount of time and resources required to manage
   vulnerability information.
  \item Facilitate long-term research and development of techniques,
   products, and processes for avoiding or mitigating vulnerabilities.\footnote{Christey \& Wysopal}
\end{enumerate}
\begin{doublespace}
To these ends Christey \& Wysopal also outline the responsibilities of both
Vendors and Reporters, which the KRACK researchers have followed perfectly.\footnotemark[6]
The Reporters initially contacted Vendors, alerting them to the
vulnerability\footnote{Cimpanu}, so that they might protect their Customers.
It is not known if the Vendors in this scenario also fulfilled their
responsibilities, but it is interesting to note that no patches were released
until after the exploit was fully and publicly disclosed.

KRACK is somewhat unique in that it is a product of academic research, but is
otherwise unremarkable in terms of its disclosure. It
was disclosed in a ``responsible'' manner that has spawned little controversy.
So what can we learn from this? First of all, it is worth
mentioning that despite its severity, the overall impact of KRACK has been low.
There have been no documented, high-profile cases of serious data being
compromised by a KRACK attack even though some Vendors took
almost a whole month to roll out patches.

KRACK was disclosed ``responsibly,'' and this would usually be cited as a
reason for the minimal impact of the vulnerability, but the timeline of
disclosure doesn't support this hypothesis. Vendors
were notified months before the paper was published, but in many cases patches
were not released until a month after the disclosure date, and some products
still have yet to be patched. The best response to the bug only occurred after
it was fully disclosed to the public. Bruce Schneier provides some insight
into this phenomenon while arguing against non-disclosure: ``To a software
company, vulnerabilities are largely an externality. That is, they affect
you -- the user -- much more than thy affect it.''\footnote{Bruce Schneier, \textit{Full Disclosure of Security Vulnerabilities a 'Damned Good Idea'}.}
He argues that to most Vendors, vulnerabilities are more of a public relations
problem than a software problem, and that Vendors usually rely on secrecy
for security rather than actually fixing their code. It is usually only after
an exploit is made public knowledge that Vendors will be motivated to fix their
broken software to avoid bad press.\footnotemark[8] This is objectively bad
policy. If an ethical hacker can discover a vulnerability, so can an unethical
one, and secrecy does nothing to protect Customers. However, the Vendors
involved in the disclosure of KRACK seem to have precisely followed this
predictive model. Though some of them were notified months before the Reporters
research was disclosed to the public, no fixes for KRACK attacks were released
until after public disclosure, and some Vendors still have not fixed the
problem months after disclosure.

Essentially, the ``responsible'' aspects of KRACK's disclosure did nothing to
mitigate its effect and its impact was minimal after full disclosure despite it
being a quite severe vulnerability, thus discounting the argument that
disclosing vulnerabilities makes Customers more at risk. The problem is that a
Reporter can act perfectly
responsibly but the whole process can be ruined by Vendors who cannot be held
accountable. With these facts taken into account, the only truly responsible
disclosure that satisfies Christey and Wysopal's goals is full disclosure.

\section*{Conclusion}
KRACK was easily the most significant vulnerability discovered in 2017.
Affecting nearly every internet user and limited only by being in range of a
wireless access point, it had the farthest-reaching scope of any such bug in
recent memory. However, it has not had nearly as large of an impact as its
potential implies. A significant mitigating factor in this and all software
exploits is the manner in which the Reporter chooses to disclose the
vulnerability, specifically the full disclosure aspect. KRACK was by all means
disclosed in a ``responsible'' manner,
but no response was enacted until the public became fully aware of the
problem. If vendors were alerted early, they should
have fixed it. They didn't, so the ``responsible'' response of the KRACK
reporters was useless. We only saw patches after full
disclosure, and this response is not at all out of the ordinary. Until
Vendors can truly be held accountable for securing their software, the only
truly responsible and ethical way to disclose vulnerabilities is fully to
the public.

\end{doublespace}
\newpage
\section*{\hfil References \hfil}
\noindent \hangindent=2em
Christey, Steve \& Chris Wysopal.
\textit{Responsible Vulnerability Disclosure Process}. \\
\url{https://tools.ietf.org/html/draft-christey-wysopal-vuln-disclosure-00}.

\noindent \hangindent=2em
Cimpanu, Catalin.
\textit{New KRACK Attack Breaks WPA2 WiFi Protocol}. \\
\url{https://www.bleepingcomputer.com/news/security/new-krack-attack-breaks-wpa2-wifi-protocol/}.

\noindent \hangindent=2em
IEEE Standards.
\textit{IEEE 802.11i-2004: Amendment 6: Medium Access Control (MAC) Security Enhancements}.
\url{http://standards.ieee.org/getieee802/download/802.11i-2004.pdf}.

\noindent \hangindent=2em
Schneier, Bruce.
\textit{Full Disclosure of Security Vulnerabilities a 'Damned Good Idea'}. \\
\url{https://www.schneier.com/essays/archives/2007/01/schneier_full_disclo.html}.

\noindent \hangindent=2em
Shepherd, Stephen A.
\textit{Vulnerability Disclosure: How do we define Responsible Disclosure}. \\
\url{https://www.sans.org/reading-room/whitepapers/threats/define-responsible-disclosure-932}.

\noindent \hangindent=2em
Vanhoef, Mathy.
\textit{Key Reinstallation Attacks Breaking WPA2 by forcing nonce reuse}. \\
\url{https://www.krackattacks.com/}.

\end{document}
