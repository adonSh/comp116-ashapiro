\documentclass[12pt]{article}
\title{KRACK and the Ethics of Disclosure}
\author{Adon Shapiro}

\usepackage[margin=1in, headheight=14.5pt]{geometry}
\usepackage{fancyhdr}
\usepackage{csquotes}
\usepackage{setspace}
\usepackage{enumitem}
\PassOptionsToPackage{hyphens}{url}
\usepackage[hidelinks]{hyperref}

\pagestyle{fancy}
\fancyhf{}
\renewcommand{\headrulewidth}{0pt}
\rhead{Shapiro \thepage}

\begin{document}
\maketitle
\section*{\hfil Abstract \hfil}
Vulnerabilities and exploits in software are discovered every day. This is a
well-known fact in the cyber-security industry. What is not so universally
agreed-upon is the proper way for an ethical hacker or security researcher to
disclose a vulnerability they have discovered, and the argument re-surfaces
with the disclosure of each new high-profile bug. The most recent such bug 
is KRACK.
In October 2017, researchers at the University of Leuven published details
of a vulnerability they had discovered in the WPA2 protocol that secures Wi-Fi
connections. They called it KRACK (Key Reinstallation AttaCK) because of the
way it takes advantage of a bug in WPA2 that allows the re-use of insecure
cryptographic keys. KRACK is significant because it is a fundamental weakness
not in a specific piece of software, but rather in the Wi-Fi standard itself,
and so everyone who uses Wi-Fi was susceptible to the attack. Because of its
recentness and its far-reaching impact, KRACK is a perfect lens through which
to view the problem of ethical disclosure.

\newpage
\begin{doublespace}
\section*{Introduction}
Disclosure, as it pertains to cyber-security, refers to the manner in which
people are made aware of vulnerabilities in their software. This can mean many
different things, however. To formalize different approaches to disclosure, it
is helpful to define three roles in the process: Vendors, Customers, and
Reporters.\footnote{Christey \& Wysopal} As defined by Christey and Wysopal,
a Vendor is ``an individual or organization who provides, develops, or
maintains software, hardware, or services, possibly for free.'' A customer is 
an ``end user of the software, hardware, or service that may be affected by the
vulnerability,'' and a Reporter is an ``individual or organization that informs
(or attempts to inform) the Vendor of the vulnerability.''\footnotemark[1] The crux of the
debate is how different approaches to disclosure affect Customers. The
``different methods'' refer to the actions and reactions of Reporters and
Vendors. At the extreme ends of the disclosure spectrum we have full disclosure
and full secrecy. Full disclosure is when a Reporter makes their discovery
completely public with little to no consideration given to the consequences of
publicizing an exploit. The reasoning behind this method is generally that the
best way to fix a vulnerability is to have it be as noticeable as possible, and
that the consequences of malicious individuals learning of the attack are
negligible.
Full secrecy is when a Reporter alerts a Vendor, but neither Reporter nor
Vendor alert the public, sometimes even after the vulnerability has been taken
care of. The rationale being that Customers will be safest if the exploit is
kept completely secret. Keeping software vulnerabilities secret, the argument goes, keeps them out of the hands of the hackers.\footnote{schneier}
Of course, these are both extremes and in reality, most
vulnerabilities are disclosed in a manner somewhere between these two
approaches. Most people try to identify a happy medium that they call
``responsible disclosure,'' but what this actually means is also a matter of
contention.

KRACK is an inherent weakness in the WPA2 protocol that
secures protected Wi-Fi networks. Essentially, the vulnerability allows an
attacker to authenticate themselves by forcing the network to reinstall an
already in-use cryptographic key (thus the name KRACK, for Key Reinstallation
AttaCK). The exploit was discovered by Belgian researchers Mathy Vanhoef and
Frank Piessens at KU Leuven in 2016 and published a paper with the results
of their findings in October 2017, however some affected vendors were notified
earlier than this.\footnote{bleepingcomputer} On the spectrum of disclosure,
this is what most people would label responsible disclosure with perhaps a bit
of a bent towards full disclosure. Though some vendors were alerted before the
bug was fully disclosed, ultimately the vulnerability was made fully
known to the public, and vendor's soon realized patches to their
implementations of WPA2.

\section*{To the Community}
So far, it may seem that KRACK is unremarkable. It was discovered by academic
researchers, disclosed in a relatively uncontroversial manner, and quickly
patched. But it is easy to overlook the severity of the exploit. For one thing,
almost anyone who uses the internet is susceptible to it. WPA2 is the standard
for secure wireless access and is relied upon for protecting almost every
modern Wi-Fi network. It is also such a recently disclosed bug that many people
are likely still susceptible to the attack if their vendor has not yet patched
their WPA2 implementation or even if the customer fails to keep their system
regularly up-to-date.

In terms of disclosure, KRACK's fresh discovery allows to view the
ramifications of its disclosure in real time, and its academic source gives it
an unusual amount of legitimacy. All told, it is a new and interesting
opportunity to gain new perspective on the old debate.

%So if this is business as usual, why do we care about KRACK? For one thing,
%because of its severity. No one is unaffected. It is also fresh and people
%are likely still vulnerable.
%
%And the disclosure question is always relevant. KRACK provides us a new and
%unique opportunity to look at the ramifications as they unfold.
%
\section*{Technical Details of the Exploit}
Since KRACK is a fundamental vulnerability in the WPA2 protocol, some
understanding of the protocol is necessary to understand the specifics of
KRACK. Wi-Fi Protected Access II (WPA2) is the name given to implementations of
the standard outlined in IEEE 802.11i-2004. The standard augments a Robust
Security Network (RSN) with the addition of a four-way Handshake and a Group
Key Handshake. First a user must provide a pre-shared keyy for initial
authentication (the Wi-Fi password). A secret Pairwise Master Key (PMK) is then
generated for the session. Next, the four-way handshake verifies that both the
access point and the client have the proper PMK without needing to disclose the
key. Once this is established, two more keys are generated: the Pairwise
Transient Key (PTK) and the Group Temporal Key (GTK).\footnote{IEEE802.11i}

KRACK exploits a vulnerability in the four-way handshake process of the
protocol. An attacker can intercepts the encrypted traffic of the handshake and
need only resend the third handshake to reset the WPA2 encryption key. Every
time the key is reset the data sent over the network will be encrypted with
the same values, allowing traffic to be gradually decrypted until the entire
key is decrypted and the network no longer secure. \footnote{krackattacks} The
attack is particularly devastating against wpa\_supplicant, the open-source
Wi-Fi client typically used on Linux operating systems. This implementation,
which is also found in many Android phones, allows the attacker to simply
install an all-zero encryption key with no need at all for the real key.

\subsection*{Defense}
Unfortunately, without a patched implementation of WPA2, there is no way to
defend against a KRACK attack. Fortunately, many vendors have released patches
that completely fix the issue. Mathy Vanhoef, the researcher who discovered the
vulnerability provides scripts to test your machine for the vulnerabiliy.

\section*{Disclosure as it Pertains to KRACK}
I said earlier that most people would consider Vanhoef's disclosure
``responsible.'' Let us now define what responsible disclosure entails. As
defined by Christey \& Wysopal, responsible disclosure aims to:
\end{doublespace}
\begin{enumerate}
  \item Ensure that vulnerabilities can be identified and eliminated
    effectively and efficiently for all parties.
  \item Minimize the risk to customers from vulnerabilities that could
   allow damage to their systems.
  \item Provide customers with sufficient information for them to evaluate
   the level of security in vendors' products.
  \item Provide the security community with the information necessary to
   develop tools and methods for identifying, managing, and reducing the
   risks of vulnerabilities in information technology.
  \item Minimize the amount of time and resources required to manage
   vulnerability information.
  \item Facilitate long-term research and development of techniques,
   products, and processes for avoiding or mitigating vulnerabilities.
\end{enumerate}
\begin{doublespace}
KRACK seems to follow this model by first notifying vendors, etc. following
Christey \& Wysopal's process outline for responsible disclosure. C \& W also
outline the responsibilities of the reporter.

KRACK is somewhat unique in that it is a product of academic research, but is
otherwise unremarkable in terms of its disclosure. Most would agree that it
was disclosed in a ``responsible'' manner. Vendors were alerted first and then
customers later. So what can we learn from this? First of all, it is worth
mentioning that despite its severity, the overall impact of KRACK has been low.
There have been no documented, high-profile cases of serious data being
compromised by a KRACK attack and many vendors (google) took almost a whole
month to roll out patches.

use other examples to explain how krack was minimal.

KRACK was disclosed ``responsibly,'' but what good did it really do. Vendors
were notified months before the paper was published, but in many cases patches
were not released until a month after the disclosure date, and some products
still have yet to be patched. Bruce Schneier.

Essentially, the ``responsible'' aspects of KRACK's disclosure did nothing to
mitigate its effect and its impact was minimal after full disclosure despite
being quite severe. The problem is that a reporter can act perfectly
responsibly but the whole thing can be ruined by vendors who cannot be held
accountable. So the vendors are useless and full disclosure really doesn't 
have that many risks.

\section*{Conclusion}
KRACK was easily the most significant vulnerability discovered in 2017.
Affecting nearly every internet user and limited only by being in range of a
wireless access point, it had the farthest-reaching scope of any such bug in
recent memory. However, it has not had nearly as large of an impact as its
potential implies. A significant mitigating factor in this and all software
exploits is the manner in which the Reporter chooses to disclose the
vulnerability. KRACK was by all means disclosed in a ``responsible'' manner,
but could have been done better. If vendors were alerted early, they should
have fixed it. They didn't, so what's the point. We only saw patches after full
disclosure. Thus, an earlier full disclosure would have been more ethical.

%KRACK's effects. Essentially KRACK was full disclosure, and its effects were
%pretty minimal. So like secrecy is bogus.

\newpage
\section*{\hfil References \hfil}
\begin{itemize}[label=]
	\item \url{https://www.krackattacks.com/}
	\item \url{https://www.schneier.com/essays/archives/2007/01/schneier_full_disclo.html}
  \item \url{https://web.archive.org/web/20060328012516/http://infosecuritymag.techtarget.com/articles/january01/columns_curmudgeons_corner.shtml}
  \item \url{https://tools.ietf.org/html/draft-christey-wysopal-vuln-disclosure-00}
  \item \url{https://www.sans.org/reading-room/whitepapers/threats/define-responsible-disclosure-932}
  \item \url{https://www.bleepingcomputer.com/news/security/new-krack-attack-breaks-wpa2-wifi-protocol/}
  \item IEEE 802.11i-2004
  \item \url{https://www.kb.cert.org/vuls/id/228519}
\end{itemize}

\end{doublespace}
\end{document}
