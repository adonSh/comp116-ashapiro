\documentclass[12pt]{article}
\title{KRACK and the Ethics of Disclosure}
\author{Adon Shapiro}

\usepackage[margin=1in]{geometry}
\usepackage[center]{titlesec}

\pagestyle{empty}

\begin{document}
\maketitle
\section*{Abstract}
Vulnerabilities and exploits in software are discovered every day. This is a
well-known fact in the cyber-security industry. What is not so universally
agreed-upon is the proper way for an ethical hacker or security researcher to
disclose a vulnerability they have discovered, and the argument re-surfaces
with the disclosure of each new high-profile bug. The most recent such bug 
is KRACK.
In October 2017, researchers at the University of Leuven published details
of a vulnerability they had discovered in the WPA2 protocol that secures Wi-Fi
connections. They called it KRACK (Key Reinstallation Attack) because of the
way it takes advantage of a bug in WPA2 that allows the re-use of insecure
cryptographic keys. KRACK is significant because it is a fundamental weakness
not in a specific piece of software, but rather in the Wi-Fi standard itself,
and so everyone who uses Wi-Fi was susceptible to the attack. Because of its
recentness and its far-reaching impact, KRACK is a perfect lens through which
to view the problem of ethical disclosure.

%Wi-Fi is susceptible, and may still be so depending on how up to
%date their software is. Of particular interest to this paper are not only the
%technical details of the exploit, but also the manner in which it was disclosed
%and how this has affected the impact of the vulnerability. Its recent-ness and
%its hu-mun-guss and far-reaching impact make KRACK a perfect candidate for this
%discussion.
%
%Earlier this year, we became aware of the KRACK vulnerability in the
%implementation of the WPA2 protocol for secure wireless internet connections.
%Essentially, (the essentials). Of additional interest to this paper,
%(disclosure). Comment about effects.
\end{document}
